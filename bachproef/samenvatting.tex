%%=============================================================================
%% Samenvatting
%%=============================================================================

% TODO: De "abstract" of samenvatting is een kernachtige (~ 1 blz. voor een
% thesis) synthese van het document.
%
% Een goede abstract biedt een kernachtig antwoord op volgende vragen:
%
% 1. Waarover gaat de bachelorproef?
% 2. Waarom heb je er over geschreven?
% 3. Hoe heb je het onderzoek uitgevoerd?
% 4. Wat waren de resultaten? Wat blijkt uit je onderzoek?
% 5. Wat betekenen je resultaten? Wat is de relevantie voor het werkveld?
%
% Daarom bestaat een abstract uit volgende componenten:
%
% - inleiding + kaderen thema
% - probleemstelling
% - (centrale) onderzoeksvraag
% - onderzoeksdoelstelling
% - methodologie
% - resultaten (beperk tot de belangrijkste, relevant voor de onderzoeksvraag)
% - conclusies, aanbevelingen, beperkingen
%
% LET OP! Een samenvatting is GEEN voorwoord!

%%---------- Nederlandse samenvatting -----------------------------------------
%
% TODO: Als je je bachelorproef in het Engels schrijft, moet je eerst een
% Nederlandse samenvatting invoegen. Haal daarvoor onderstaande code uit
% commentaar.
% Wie zijn bachelorproef in het Nederlands schrijft, kan dit negeren, de inhoud
% wordt niet in het document ingevoegd.

\IfLanguageName{english}{%
\selectlanguage{dutch}
\chapter*{Samenvatting}
\selectlanguage{english}
}{}

%%---------- Samenvatting -----------------------------------------------------
% De samenvatting in de hoofdtaal van het document

\chapter*{\IfLanguageName{dutch}{Samenvatting}{Abstract}}

Deze paper onderzoekt de mogelijkheid van het gebruik van AI om nieuwsverhalen om te zetten in visuele kunst. De focus ligt op de vraag in hoeverre AI effectief de essentie van een nieuwsverhaal kan vastleggen en vertalen naar visuele kunst. Daarnaast wordt onderzocht hoe technologieën zoals GPT (Generative Pre-trained Transformer) en DALL-E kunnen worden toegepast in de context van kunst, communicatie en maatschappij.\\

De onderzoeksvraag wordt beantwoord door middel van een combinatie van webscraping, waarbij nieuwsartikelen worden verzameld, en het gebruik van een GPT-model om de essentie van de nieuwsverhalen vast te leggen. Vervolgens wordt DALL-E ingezet om op basis van de gegenereerde prompts visuele kunstwerken te creëren.\\

De resultaten tonen aan dat AI in staat is om in zekere mate de essentie van een nieuwsverhaal om te zetten naar visuele kunst. Het juist formuleren van prompts en het voeren van een dialoog met het GPT-model zijn belangrijke aspecten voor het verkrijgen van gewenste resultaten. Echter, vanwege de abstracte aard van kunst en de subjectieve interpretatie ervan, is het niet altijd mogelijk om de boodschap van het nieuwsverhaal direct herkenbaar te maken in het gegenereerde kunstwerk.\\

Hoewel AI geavanceerd genoeg is om kunstwerken te genereren op basis van nieuwsverhalen, zijn er uitdagingen en beperkingen verbonden aan het proces. Het trainen van modellen met specifieke corpus van nieuwsartikelen kan helpen om betere resultaten te behalen. Webscraping is een effectieve methode om nieuwsartikelen te verzamelen, maar vereist regelmatige aanpassingen vanwege veranderingen in de structuur en lay-out van websites.\\

In conclusie biedt dit onderzoek inzicht in de mogelijkheden en beperkingen van AI bij het omzetten van nieuwsverhalen naar visuele kunst. Het benadrukt de waarde van de juiste formulering van prompts, de dialoog met het model en het begrip van de abstracte aard van kunst. Het biedt ook aanknopingspunten voor verdere onderzoeken naar optimalisatiestrategieën en toepassingen van AI in de kunst en communicatiesector.

\pagebreak
