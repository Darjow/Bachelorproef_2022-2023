%%=============================================================================
%% Conclusie
%%=============================================================================

\chapter{Conclusie}%
\label{ch:conclusie}
In dit onderzoek werd een antwoord gegeven op de hoofdonderzoeksvraag en deelvragen. En lichten we kort toe over mogelijke toekomstige onderzoeksmogelijkheden.  \\

We beginnen met een antwoord te bieden op de deelvragen omdat deze een betere context en inzicht zullen geven voor de onderzoeksvraag. 

\section{\IfLanguageName{dutch}{Deelvraag: In welke mate kan AI effectief de essentie van een nieuwsverhaal vastleggen en vertalen naar visuele kunst?}{Sub-question: To what extent can AI effectively capture the essence of a news story and translate it into visual art?}}%

In mijn mening en wat blijkt uit de turingtest kan AI in zekere mate effectief de essentie van een nieuwsverhaal vastleggen en vertalen naar visuele kunst. Het belangrijkste aspect is het gebruik van de juiste prompts bij GPT om de gewenste resultaten te verkrijgen. Het komt echter voor dat tijdens de conversatie GPT niet altijd in staat is om het hoogtepunt van het nieuwsverhaal weer te geven waar de kenmerken duidelijk zichtbaar zijn. Daarom is het van cruciaal belang om goede prompts te gebruiken en de conversatie met het model aan te gaan. \\

Daarnaast moet worden opgemerkt dat kunst abstract is en dat iedereen kunst anders interpreteert. In sommige gevallen kan GPT een zeer abstract kunstwerk genereren, waardoor de boodschap van het nieuwsverhaal niet direct herkenbaar is. Dit kan invloed hebben op de mate waarin AI in staat is om de essentie van het nieuwsverhaal accuraat over te brengen in visuele kunst. \\

Het is ook belangrijk om aan te vullen dat de resultaten van AI-gegenereerde kunstwerken sterk afhankelijk zijn van de trainingsgegevens en het model zelf. Het is mogelijk dat AI bepaalde nuances of complexe aspecten van een nieuwsverhaal niet volledig kan vastleggen, omdat het model alleen kan putten uit de informatie waarmee het is getraind. \\

Kortom, hoewel AI in staat is om de essentie van een nieuwsverhaal te vertalen naar visuele kunst, zijn er uitdagingen en beperkingen verbonden aan het proces. Het juist formuleren van prompts, het voeren van een dialoog met het model en het begrijpen van de abstracte aard van kunst kunnen bijdragen aan het vergroten van de effectiviteit van AI bij het vertalen van nieuwsverhalen naar visuele kunst.
    
\section{\IfLanguageName{dutch}{Deelvraag: Hoe kunnen technologieën zoals GPT en DALL-E worden toegepast in de context van kunst, communicatie en maatschappij? }{Sub-question: How can technologies such as GPT and DALL-E be applied in the context of art, communication and society?}}%

Zoals wellicht duidelijk is uit deze paper zou het in eerste instantie gebruikt kunnen worden om kunst te genereren in een krant of nieuwswebsite. Zo kun je bij elk artikel een kunstwerk laten genereren , of eventueel per thema het hoogtepunt gaan kenmerken aan de hand van een schilderij.   \\

In de context van communicatie kunnen GPT en DALL-E worden gebruikt voor het genereren van content. Bijvoorbeeld, marketing- en reclamebureaus kunnen GPT gebruiken om aantrekkelijke advertentieteksten te creëren. Journalisten kunnen GPT gebruiken om nieuwsartikelen te schrijven of samenvattingen te genereren. \\

Tenslotte kan het gebruikt worden voor het analyseren van grote hoeveelheden tekstuele en visuele gegevens om inzichten te verkrijgen over maatschappelijke problemen en trends.


\section{\IfLanguageName{dutch}{Deelvraag: Hoe kan webscraping gebruikt worden om nieuwsartikelen te kunnen scrapen?}{Sub-question: How can web scraping be used to scrape news articles?}  }%
Om webscraping effectief toe te passen, is het belangrijk om de DOM-structuur van de nieuwswebsites te analyseren. Hierbij onderzoek je de HTML-elementen, tags en klassen die de relevante informatie bevatten, zoals de titels, inhoud en auteurs van de nieuwsartikelen. Door patronen te herkennen of consistente structuren te identificeren, kun je een aanpak ontwikkelen om deze informatie te extraheren. \\

Met behulp van de gekozen tool kun je nu de webscraper gaan implementeren. Hiermee stel je HTTP-verzoeken in naar de doelwebsite(s), extraheer je de relevante gegevens met behulp van CSS-selectors, en sla je de geëxtraheerde informatie op in een geschikt formaat.  \\

Het is belangrijk om op te merken dat websites regelmatig hun structuur en lay-out kunnen wijzigen, wat gevolgen kan hebben voor je scraper. Daarom is het goed om je scraper op een manier te implementeren dat hij minder specifiek is, maar veralgemeend.

\section{\IfLanguageName{dutch}{Onderzoeksvraag: Is AI geavanceerd genoeg om kunstwerken te genereren waarvan de boodschap herkenbaar is uit het dagelijks nieuws?}{Research question: Is AI advanced enough to generate works of art whose message is recognizable from the daily news?}  }%
Op basis van voorgaande deelvragen kunnen we besluiten dat met de opkomst van AI-technologieën  het mogelijk is geworden om kunstwerken te genereren die gebaseerd zijn op nieuwsartikelen en/of andere tekstbronnen. \\

De kunstwerken kunnen de essentie van het nieuwsverhaal vastleggen en vertalen naar visuele vormen, maar er zijn uitdagingen en beperkingen verbonden aan dit proces. Het gebruik van de juiste prompts bij GPT en het begrijpen van de abstracte aard van kunst zijn essentiële factoren om de effectiviteit van AI te vergroten bij het vertalen van nieuwsverhalen naar visuele kunst. Bovendien kunnen technologieën zoals GPT en DALL-E breder worden toegepast in de context van kunst, communicatie en maatschappij. Ze kunnen bijvoorbeeld gebruikt worden voor het genereren van content, zoals advertentieteksten, nieuwsartikelen of andere zaken.


\section{\IfLanguageName{dutch}{Verdere onderzoeksmogelijkheden}{Further research opportunities}  }%
De resultaten van dit onderzoek hebben inzichten opgeleverd met betrekking tot de mogelijkheden en uitdagingen van AI bij het genereren van kunstwerken op basis van nieuwsverhalen. Niettemin zijn er nog verschillende interessante vragen en onderzoeksperspectieven die verder kunnen worden verkend. Zo kan er toekomstig onderzoek worden gedaan naar de mate van accuraatheid en coherentie van AI-modellen bij het overbrengen van de boodschap van nieuwsverhalen in visuele kunst. Daarnaast kan er gekeken worden naar de integratie van abstracte aspecten van kunst in AI-algoritmen om meer expressieve en interpretatieve kunstwerken te genereren.  Daarnaast kunnen ook de mogelijkheden van AI-gestuurde kunstwerken in de journalistiek en communicatie verder worden onderzocht, evenals de rol van menselijke input in dit proces.

% TODO: Trek een duidelijke conclusie, in de vorm van een antwoord op de
% onderzoeksvra(a)g(en). Wat was jouw bijdrage aan het onderzoeksdomein en
% hoe biedt dit meerwaarde aan het vakgebied/doelgroep? 
% Reflecteer kritisch over het resultaat. In Engelse teksten wordt deze sectie
% ``Discussion'' genoemd. Had je deze uitkomst verwacht? Zijn er zaken die nog
% niet duidelijk zijn?
% Heeft het onderzoek geleid tot nieuwe vragen die uitnodigen tot verder 
%onderzoek?


