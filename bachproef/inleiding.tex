%%=============================================================================
%% Inleiding
%%=============================================================================

\chapter{\IfLanguageName{dutch}{Inleiding}{Introduction}}%
\label{ch:inleiding}

\noindent
In een wereld waarin technologie voortdurend evolueert, rijst de vraag of generatieve kunstmatige intelligentie (AI) al geavanceerd genoeg is om betekenisvolle creaties te genereren die een boodschap overbrengen op basis van dagelijks nieuws. Deze bachelorproef onderzoekt de huidige stand van generatieve AI en beoordeelt de effectiviteit en kwaliteit van AI-generaties bij het communiceren van nieuwsgerelateerde boodschappen op unieke en boeiende wijze met het publiek. Door de combinatie van geavanceerde AI-technieken en automatisering van webscraping, verkennen we de grenzen van de interactie tussen technologie en informatievoorziening, en bepalen we of generatieve AI al voldoende ontwikkeld is om deze uitdaging aan te gaan.
\pagebreak


\section{\IfLanguageName{dutch}{Context en achtergrond}{Context and background}}%
\label{sec:context}
\noindent
In de afgelopen jaren heeft kunstmatige intelligentie (AI) aanzienlijke vooruitgang geboekt, met name op het gebied van generatieve modellen. Deze vooruitgang heeft geresulteerd in geavanceerde systemen die in staat zijn om zelfstandig creatieve output te genereren. Het potentieel van deze technologieën om kunst en communicatie te transformeren is enorm. Hieronder volgt een overzicht van de belangrijkste ontwikkelingen en trends op het gebied van generatieve AI. \\

\noindent
De relevantie van dit onderzoek ligt in de mogelijkheid om een brug te slaan tussen kunst, technologie en maatschappij. Het kan leiden tot nieuwe manieren van nieuwscommunicatie en artistieke expressie. De methodologie omvat het gebruik van AI-algoritmen en webscraping om kunstwerken te genereren op basis van actuele gebeurtenissen.


\section{\IfLanguageName{dutch}{Probleemstelling}{Problem Statement}}%
\label{sec:probleemstelling}
\noindent
Het is onduidelijk of AI geavanceerd genoeg is om kunstwerken te creëren die een nieuwsgerelateerde boodschap effectief overbrengen. Dit onderzoek zal deze vraag beantwoorden en de praktische toepassingen van dergelijke technologieën verkennen voor een doelgroep van kunstenaars, nieuwsorganisaties en ontwikkelaars.

\section{\IfLanguageName{dutch}{Onderzoeksvraag}{Research question}}%
\label{sec:onderzoeksvraag}

\noindent
Kan AI geavanceerde kunstwerken genereren die een boodschap uit het dagelijks nieuws effectief overbrengen, en zo ja, hoe kunnen deze technologieën worden toegepast in de context van kunst, communicatie en maatschappij?

Deelvragen zijn onder andere:

\begin{itemize}
    \item Welke generatieve AI-technieken zijn het meest geschikt voor het creëren van nieuwsgerelateerde kunstwerken?
    \item Hoe kunnen webscraping-technieken gebruiken om nieuwsartikelen te kunnen scrapen?
    \item In welke mate kan AI effectief de essentie van een nieuwsverhaal vastleggen en vertalen naar visuele kunst?
\end{itemize}

\pagebreak

\section{\IfLanguageName{dutch}{Onderzoeksdoelstelling}{Research objective}}%
\label{sec:onderzoeksdoelstelling}

\noindent
Het doel van deze bachelorproef is om de mogelijkheden van generatieve AI te onderzoeken bij het genereren van kunstwerken die nieuwsgerelateerde boodschappen communiceren. Hierbij zal er specifiek worden onderzocht of generatieve AI al geavanceerd genoeg is om dergelijke kunstwerken te creëren en welke technieken hiervoor het meest geschikt zijn.  \\

Deze bachelorproef zal zich richten op kunstenaars, onderzoekers en bedrijven die geïnteresseerd zijn in de mogelijkheden van generatieve AI voor het creëren van betekenisvolle kunstwerken. Het uiteindelijke doel is om een aanbeveling te doen over de haalbaarheid en effectiviteit van generatieve AI voor het communiceren van nieuwsgerelateerde boodschappen door middel van kunst.

\section{\IfLanguageName{dutch}{Opzet van deze bachelorproef}{Structure of this bachelor thesis}}%
\label{sec:opzet-bachelorproef}

% Het is gebruikelijk aan het einde van de inleiding een overzicht te
% geven van de opbouw van de rest van de tekst. Deze sectie bevat al een aanzet
% die je kan aanvullen/aanpassen in functie van je eigen tekst.

De rest van deze bachelorproef is als volgt opgebouwd: \\

In Hoofdstuk~\ref{ch:stand-van-zaken} wordt een overzicht gegeven van de stand van zaken binnen het onderzoeksdomein, op basis van een literatuurstudie. \\
Hoofdstuk~\ref{ch:methodologie} belicht de gebruikte methodologie en onderzoekstechnieken die zijn toegepast om de onderzoeksvragen te beantwoorden.
Hoofdstuk~\ref{ch:proof-of-concept} biedt een gedetailleerde uitleg over de implementatie van de applicatie, waarbij stap voor stap de ontwikkeling wordt besproken.
Het evaluatieproces, beschreven in Hoofdstuk~\ref{ch:evaluatieproces}, heeft als doel de geïnterpreteerde waarde van het schilderij te achterhalen. Dit proces omvat een turingtest, waarbij verschillende persona 3 samples krijgen aangeboden die afkomstig zijn van verschillende dagen. De deelnemers worden uitgedaagd om een interpretatie van het schilderij te geven en deze te koppelen aan een specifiek artikel.

Tot slot wordt in Hoofdstuk~\ref{ch:conclusie} de conclusie gepresenteerd en worden de onderzoeksvragen beantwoord.