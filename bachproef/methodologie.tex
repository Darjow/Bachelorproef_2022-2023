%%=============================================================================
%% Methodologie
%%=============================================================================

\chapter{\IfLanguageName{dutch}{Methodologie}{Methodology}}%
\label{ch:methodologie}

%% TODO: Hoe ben je te werk gegaan? Verdeel je onderzoek in grote fasen, en
%% licht in elke fase toe welke stappen je gevolgd hebt. Verantwoord waarom je
%% op deze manier te werk gegaan bent. Je moet kunnen aantonen dat je de best
%% mogelijke manier toegepast hebt om een antwoord te vinden op de
%% onderzoeksvraag.
\section{Inleiding}
Het onderzoek begint met een grondige literatuurstudie die te vinden is in Hoofdstuk 2. Deze literatuurstudie bepaalt de onderwerpen waarop dit onderzoek zich zal richten en onderzoekt enkele belangrijke en veelvoorkomende webscraping- en AI-modellen die momenteel beschikbaar zijn.  \\

\section{Fases}
\begin{itemize}
    \item \textbf{Fase 1}: Ontwikkelen van een webscraper die in staat is om de website, titel op een correct en gestructureerde manier te verzamelen. 
    \item \textbf{Fase 2}: Gebruik van verzamelde artikelen als input voor het GPT-model, dat prompts genereert die dienen als basis voor het genereren van schilderijen met behulp van OpenAI's API.
    \item \textbf{Fase 3}: Fenereren van een kunstwerk met DALL-E op basis van de prompt uit vorige fase 2.
\end{itemize} 

\section{Onderverdeling}

Hoofdstuk~\ref{ch:proof-of-concept} Proof-of-concept

Dit hoofdstuk beschrijft de proof-of-concept van het onderzoek. Het bevat de ontwikkeling van de webscraper in fase 1 en het gebruik van de verzamelde artikelen als input voor het GPT en DALL-E model in fase 2 en 3. Er zal uitgelegd worden hoe de gegenereerde prompts dienen als basis voor het genereren van schilderijen met behulp van DALL-E. Het hoofddoel van dit hoofdstuk is om te laten zien dat het haalbaar is om kunstwerken te genereren met behulp van webscraping, AI en DALL-E. \\

Hoofdstuk~\ref{ch:evaluatieproces} Evaluatieproces

Dit hoofdstuk beschrijft het evaluatieproces van het gegenereerde kunstwerk. In fase 4 wordt er een evaluatieproces beschreven dat de te geïnterpreteerde waarde van het schilderij achterhaalt. Dit zal gedaan worden door een belangrijke datum uit het korte verleden te scrapen en te zien of de kernboodschap voor deze belangrijke data naar boven komt. 



