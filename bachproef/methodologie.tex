%%=============================================================================
%% Methodologie
%%=============================================================================

\chapter{\IfLanguageName{dutch}{Methodologie}{Methodology}}%
\label{ch:methodologie}

%% TODO: Hoe ben je te werk gegaan? Verdeel je onderzoek in grote fasen, en
%% licht in elke fase toe welke stappen je gevolgd hebt. Verantwoord waarom je
%% op deze manier te werk gegaan bent. Je moet kunnen aantonen dat je de best
%% mogelijke manier toegepast hebt om een antwoord te vinden op de
%% onderzoeksvraag.
\section{Inleiding}
Het onderzoek begint met een grondige literatuurstudie die te vinden is in Hoofdstuk 2. Deze literatuurstudie bepaalt de onderwerpen waarop dit onderzoek zich zal richten en onderzoekt enkele belangrijke en veelvoorkomende webscraping- en AI-modellen die momenteel beschikbaar zijn.  \\

\section{Fases}
\begin{itemize}
    \item \textbf{Fase 1}: Ontwikkelen van een webscraper die in staat is om de website, titel op een correct en gestructureerde manier te verzamelen. 
    \item \textbf{Fase 2}: Prompt genereren met GPT met behulp van de verzamelde artikels. 
    \item \textbf{Fase 3}: Genereren van een kunstwerk met DALL-E op basis van de prompt uit vorige fase 2.
\end{itemize} 

\section{Onderverdeling}

Hoofdstuk \ref{ch:proof-of-concept}  beschrijft de proof-of-concept van het onderzoek. Het bevat de ontwikkeling van de webscraper in fase 1 en het gebruik van de verzamelde artikelen als input voor het GPT en DALL-E model in fase 2 en 3. Er zal uitgelegd worden hoe de gegenereerde prompts dienen als basis voor het genereren van schilderijen met behulp van DALL-E. Het doel van dit hoofdstuk is om de implementatie en werking van de applicatie aan te tonen op basis van de verkregen informatie uit hoofdstuk \ref{ch:stand-van-zaken}.  \\

Hoofdstuk beschrijft \ref{ch:evaluatieproces} de evaluatieproces van het gegenereerde kunstwerk. In dit hoofdstuk wordt er een evaluatieproces beschreven dat de te geïnterpreteerde waarde van het schilderij achterhaalt. Dit zal gedaan worden aan de hand van een turingtest waar we de deelnemers zullen bevragen om een boodschap uit een schilderij te halen. (of een nieuwsartikel hieraan te koppelen) Met behulp van deze turingtest kunnen we bepalen of AI in staat is om de boodschap van het nieuws over te brengen.


