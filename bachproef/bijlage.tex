\section{Resultaat scraper}
\begin{listing}
    \begin{minted}[breaklines]{json}
        {
            "https://www.hln.be": [
            "Nederlander (37) overlijdt na ongeval met ontplofte petanquebal tijdens vrijgezellenfeest in Stavelot",
            "EXCLUSIEVE ANALYSE. “Oekraïne heeft 4 mirakels nodig”",
            "Staatssecretaris Bertrand spreekt straffe taal tegen banken: “Maar de overheid moet zelf ook het goede voorbeeld geven”",
            "“Russisch leger rekruteerde dit jaar al bijna 120.000 soldaten”",
            "Barack Obama niet meer welkom in Rusland: inreisverbod tegen voormalige president en 499 andere Amerikanen",
            "Opmars Aziatische hoornaar niet meer te stuiten in Vlaanderen: \"Kwestie van tijd voor er doden vallen”",
            "Nederlander (37) overlijdt na ongeval met ontplofte petanquebal tijdens vrijgezellenfeest in Stavelot",
            "Poetins geheime bunkercomplex onder gigantisch paleis aan Zwarte Zee onthuld: zo ziet het eruit"
            ],
            "https://www.demorgen.be": [
            "Voor de mensen achteraan in de zaal: het is absoluut zeker dat de klimaatverandering tot meer extreme weersituaties leidt",
            "Pedagoog Pedro De Bruyckere: ‘Leerlingen in Engeland en Ierland zitten ook constant op hun smartphone, en zij gingen erop vooruit’",
            "Transgender activiste Jenna Boeve: ‘Als ik in lingerie naar de Pride ga, is dat ook een vorm van protest’",
            "Waarom meer geliefden hebben (niet) altijd beter is: ‘Fantastisch om je partner te vertellen dat je verliefd bent op een ander’",
            "Jane Fonda: ‘Franse regisseur vroeg me om met hem naar bed te gaan ‘om te zien hoe mijn orgasmes waren’ voor een rol’",
            "Enorme gewicht van wolkenkrabbers brengt New York in problemen bij verwachte zeespiegelstijging",
            "Rubio wint bergrit van 74 kilometer, wapenstilstand onder favorieten",
            "Britse telecomreus vervangt 10.000 jobs door AI",
            "‘VS zetten licht op groen voor training Oekraïense piloten in westerse gevechtsvliegtuigen’",
            "Live - Oekraïne. Rusland vaardigt arrestatiebevel uit tegen aanklager Internationaal Strafhof die Poetin wil laten arresteren",
            "‘Helft leerlingen secundair onderwijs heeft moeite met lezen’",
            "Erdogan prijst ‘speciale relatie’ met Poetin: ‘Rusland en Turkije hebben elkaar nodig op alle vlakken’",
            "‘Ook aan hun kant ontploft er heel wat’: Oekraïne lijkt te dralen met zijn tegenoffensief",
            "‘VS zetten licht op groen voor training Oekraïense piloten in westerse gevechtsvliegtuigen’"
            ]
        }
        
    \end{minted}
    \label{bijlage:json-input}
    \captionof{figure}{De gescrapete json-data op vrijdag 19 mei 20:07}
\end{listing}
\pagebreak


\section{Implementatie Scraper op specifieke dag}
\label{bijlage:changes_scraper}
\subsection{Aanpassing gesprek GPT}
Omdat we nu op een specifieke dag scrapen, kunnen we dit ook meegeven aan GPT. Verder doen we kleine aanpassingen aan de conversatie voor betere resultaten. Hieronder vindt je de nieuwe implementatie van GPT. 
\begin{pythoncode}{../../../workspace/paper/gpt2.0.py}
\end{pythoncode}

\subsection{Implementatie scraper}
Om te kunnen scrapen op een specifieke dag, en niet uren willen wachten tot de scraper alle artikels van een website heeft overlopen, scrapen we nu het archief van de Standaard. In volgende code-fragment vind je de geïmplementeerde scraper. 
\begin{pythoncode}{../../../workspace/paper/StandaardScraper.py}
    Merk op de scraper gebruikt nu HTMLSession() om de DOM te parsen, dit is omdat met bs4 het niet mogelijk was. 
\end{pythoncode}

\subsection{Aanpassingen \_\_main\_\_.py}
Uiteraard moeten we nu ook de \_\_main\_\_.py aanpassen zodat hij deze wijzigingen uitvoerd. In volgende code-fragment zie je de wijzigingen. 

\begin{pythoncode}{../../../workspace/paper/newMain.py}

\end{pythoncode}

\section{Resultaten Turingtest}
\label{bijlage:result_turingtest}
\subsection{Persona: Ronan}
Ronan is een creatieve ICT-student die graag game development wil studeren. Hij is een getalenteerde tekenaar, maar hij volgt geen nieuws. Laten we eens kijken naar de bevindingen van Ronan op basis van de samples en ontdekken of hij de boodschap heeft kunnen achterhalen.

\subsubsection{Sample 1}
Bij het bekijken van het schilderij merkt Ronan op: "Dat in het midden is niet echt duidelijk, maar het lijkt erop dat er verschillende landen iets willen doen in Oekraïne." \\

Ronan lijkt de kernboodschap van dit schilderij matig te hebben begrepen. Hij heeft opgemerkt dat er verschillende landen betrokken zijn bij een kwestie in Oekraïne. Hoewel hij niet specifiek weet welk artikel hieraan ten grondslag ligt, heeft hij de basis toch begrepen.  \\

Ronan vertelde me dat als de vlaggen die centraal worden afgebeeld een betekenis hadden, indien hij wist waarvoor de vlaggen stonden kon hij waarschijnlijk de kernboodschap hieruit afleiden. 

\subsubsection{Sample 2}
Bij het analyseren van dit schilderij geeft Ronan zijn interpretatie: "Iets met medicatie, een nieuw medicijn en/of anticonceptiemiddel? Wat de vrouw in haar hand houdt, kan misschien een nieuw geneesmiddel zijn dat op de markt is gelanceerd." \\

Ronan heeft de boodschap van dit schilderij opgepikt en vermoedt dat het gaat over medicatie. Hij suggereert dat er mogelijk een nieuw medicijn of anticonceptiemiddel is ontwikkeld, en de vrouw in het schilderij houdt misschien het nieuwe geneesmiddel vast. Hoewel hij niet precies weet welk artikel hieraan gerelateerd is, heeft hij een algemeen begrip van het onderwerp. \\

\subsubsection{Sample 3}
"Ik zie iets met Europa, een duim omhoog wat duidt op goedkeuring. Het lijkt op een computerscherm, dus misschien heeft het iets te maken met een website. Ik denk dat het Europees Parlement een nieuwe wet heeft goedgekeurd die te maken heeft met regels op het internet." \\

Ronan heeft de boodschap niet kunnen achterhalen. Hij heeft wel opgemerkt dat de EU duidelijk aanwezig is in het schilderij, maar hij herkende de duimpjes niet als de iconische "likes" op Facebook. Na het tonen van het artikel zei Ronan dat het mogelijk was geweest om dit te herkennen, maar dat abstractie correct geïnterpreteerd moet kunnen worden.

\subsubsection{Sample 4}
In Sample 4 heeft Ronan zonder enige twijfel verteld dat dit de aanslagen op de bekende Twin Towers weerspiegeld. Zijn directe herkenning van deze gebeurtenis toont aan dat het schilderij een duidelijk visueel verband had met een historisch moment dat wereldwijd bekend is. \\

De gebeurtenis van deze schilderij werd meteen achterhaald.

\subsubsection{Sample 5}
In Sample 5 twijfelde Ronan even voordat hij bevestigde dat het schilderij de Notre-Dame toonde tijdens de verwoestende brand. Hoewel er een kort moment van aarzeling was, slaagde Ronan erin om de gebeurtenis correct te identificeren. Dit laat zien dat het schilderij voldoende visuele aanwijzingen bevatte om de specifieke gebeurtenis te herkennen, zelfs als er enige interpretatie of herinnering nodig was. \\

Deze sample bewijst dat het mogelijk is om betekenisvolle gebeurtenissen via kunstwerken te onthullen. 


\subsection{Persona: Arthur}
Arthur is een laatstejaarsstudent Informatica met een matige kennis van kunst. Hoewel hij zelf niet actief betrokken is bij het creëren van kunst, heeft hij een goed begrip van het onderwerp. Arthur volgt het nieuws niet. 

\subsubsection{Sample 1}
Arthur merkte het volgende op: ``Dit gaat gegarandeerd over Oekraïne. Het lijkt erop aan de hand van de vlag dat een groep landen samenwerken binnen Oekraïne voor één bepaald doel.'' \\

We besluiten dat arthur de boodschap van dit schilderij goed begrepen heeft en concludeert dat het gaat over samenwerking tussen landen in Oekraïne. Hoewel hij niet specifiek weet welk artikel hieraan gerelateerd is, heeft hij de essentie begrepen. 

\subsubsection{Sample 2}
Arthur merkte het volgende op: "Op het eerste gezicht lijkt het alsof een vrouw in dit schilderij lijdt. Ik zie ook medicatie haar liggen. Zou dit kunnen gaan over de nieuwe medicatie voor vrouwen die in hun menopauze zitten?" \\

Arthur heeft de boodschap van dit schilderij opgepikt en vermoedt dat het te maken heeft met het lijden van vrouwen. Hij maakt de associatie met medicatie en suggereert dat het mogelijk gaat over nieuwe medicatie voor vrouwen in hun menopauze. 

\subsubsection{Sample 3}
Arthur had de volgende interpretatie: ``Iets met europese unie, iets dat goedgekeurd moet worden of al is gebeurd?'' \\

Arthur heeft de boodschap van dit schilderij niet specifiek kunnen achterhalen, maar hij herkent de aanwezigheid van de Europese Unie. Hij vermoedt dat het schilderij mogelijk te maken heeft met goedkeuring van iets, ofwel iets dat nog moet gebeuren of al heeft plaatsgevonden. Doordat hij geen specifieke details of context kan geven, heeft hij de boodschap niet kunnen achterhalen. 

\subsubsection{Sample 4}
Bij het bekijken van het schilderij moest Arthur toch even nadenken. Na het opmerken van het vliegtuig, maakte hij de gok dat het verband hield met de gebeurtenis van 9/11. Het bleek een goede gok te zijn, aangezien zijn interpretatie juist was.

\subsubsection{Sample 5}
Bij het aanschouwen van het schilderij kon Arthur niet direct de specifieke gebeurtenis identificeren, en beschreef hij het simpelweg als "een brandende kerk aan het water". Pas nadat ik hem vertelde dat het de afbeelding van de Notre-Dame in Parijs betrof, vond hij het schilderij zeer duidelijk. Het bleek dat hij gewoon niet aan die specifieke gebeurtenis had gedacht, maar de boodschap van het schilderij kwam alsnog goed bij hem over.

\subsection{Persona: Arjan}
Arjan is geboren in Turkije en opgegroeid in Nederland. In zijn jongere jaren heeft hij zich veel beziggehouden met het ontwerpen van logo's en banners voor profielen, wat zijn vaardigheden op het gebied van grafisch ontwerp versterkt. Dit maakt hem een interessante testpersona voor de turingtest, aangezien zijn ervaring en kennis een waardevolle troef kunnen zijn. Arjan volgt enkel de hoogtepunten.

\subsubsection{Sample 1}
``Ik kan dit artikel niet helemaal plaatsen. Hoewel het over Oekraïne gaat, kan ik de boodschap er niet uit halen. Als ik zou moeten gokken, zou ik zeggen dat er meerdere landen bij betrokken zijn vanwege de aanwezigheid van vlaggen. Maar ik kan de precieze boodschap niet achterhalen.'' \\

Net als de vorige persona kan Arjan ook concluderen dat het artikel over Oekraïne gaat en dat andere landen mogelijk betrokken zijn. De boodschap blijft echter onduidelijk.

\subsubsection{Sample 2}
Arjan vermoedde dat dit kunstwerk ging over een vrouw die verslaafd is aan medicatie. Echter, toen ik hem vertelde dat het eigenlijk ging over een nieuwe medicatie, kon hij meteen zeggen dat het specifiek ging over medicatie voor de menopauze. \\

De oorspronkelijke boodschap werd dus verkeerd geïnterpreteerd door Arjan. Echter, met een hint kon hij wel de juiste boodschap interpreteren.

\subsubsection{Sample 3}
"Europese Unie, televisie, duimen. Heeft dit misschien iets te maken met Facebook? Gaat dit over de boete die de Europese Unie heeft opgelegd aan Meta?" \\

Arjan heeft de mogelijke connectie gelegd tussen de Europese Unie, televisie en duimen, en vermoedt dat het mogelijk verband houdt met Facebook. Hij suggereert dat het specifiek zou kunnen gaan over de boete die de Europese Unie heeft opgelegd aan Meta. \\

Arjan heeft successvol de boodschap hieruit gehaald en ook de correcte artikel hieraan gelinkt. 

\subsubsection{Sample 4}
Bij het bekijken van de schilderij, herkende Arjan onmiddellijk dat de Twin Towers het centrale onderwerp waren. Aan de hand van de de vliegtuigen op het schilderij legt Arjan direct het verband met de gebeurtenissen van 9/11. \\

 Zijn interpretatie was correct en hij twijfelde niet over de boodschap die het schilderij overbracht.

\subsubsection{Sample 5}
Arjan had geen enkel probleem met hier de betekenis eruit te halen. Hij kon duidelijk zeggen dat dit over de Notre-Dame gaat in Parijs die enkele jaren geleden in brand stond. 

\subsection{Persona: Lucas}
Lucas is een gepassioneerde software engineer die in zijn vrije tijd graag muziek draait. Hoewel hij geen specifieke artistieke achtergrond heeft, is hij wel creatief in zijn vakgebied. Lucas volgt echter het nieuws niet.
\subsubsection{Sample 1}
Bij het bekijken van het schilderij herkende Lucas een gevoel van patriotisme en merkte hij op dat de rode lijn in Oekraïne mogelijk een splitsing of een grens vertegenwoordigde. Hij kon echter geen verband leggen tussen de vlaggen en meerdere landen. Met andere woorden, Lucas kon de exacte boodschap van dit schilderij niet achterhalen.
\subsubsection{Sample 2}
Het wazige beeld en de afbeelding van medicatie deden Lucas denken aan een verslaving. Toen ik hem vertelde dat het schilderij eigenlijk ging over een nieuw medicijn voor vrouwen, kon Lucas de link leggen en begrijpen waar het over ging.
\subsubsection{Sample 3}
Lucas interpreteerde dit schilderij als iets waarover gestemd moest worden. De mensen die hun duim omhoog staken, suggereerden dat er een nieuwe regel of wet werd beoordeeld. Nadat ik hem vertelde dat de duimen eigenlijk het Facebook-symbool waren, kon hij de link leggen, maar slaagde hij er nog steeds niet in om de exacte boodschap eruit te halen.
\subsubsection{Sample 4}
De boodschap van deze sample kwam duidelijk over bij Lucas. Hoewel hij aanvankelijk een kleine twijfel had, kon hij uiteindelijk bevestigen dat het schilderij betrekking had op de gebeurtenissen van 9/11. Zijn interpretatie was helder en de boodschap kwam zeker goed over.

\subsubsection{Sample 5}
Lucas herkende direct dat het schilderij de Notre-Dame in Parijs afbeeldde. Hij verklaarde: "Omdat deze gebeurtenis nog vers in het geheugen zit, kon ik het waarschijnlijk gemakkelijk herkennen."

\subsection{Persona: Gustas}
Gustas is een Litouwse man met een sterke interesse in automatisering. Hij heeft zijn studies stopgezet om zijn eigen bedrijf op te richten. Gustas heeft gedurende 4 jaar gestudeerd aan een kunstschool. Vanwege zijn nationaliteit en achtergrond in de kunst lijkt deze persona zeer interessant voor de turingtest. Het feit dat Gustas het nieuws niet volgt, voegt een interessant perspectief toe aan zijn beoordeling van de kunstwerken.

\subsubsection{Sample 1}
Bij het eerste voorbeeld merkte Gustas de volgende boodschap op: "Ik denk dat het nieuws gaat over Oekraïne die een deel van het land terugneemt?" Helaas was dit niet de juiste boodschap. Gustas gaf aan dat de rode lijn in het kunstwerk deed vermoeden dat het een bezet gebied van Rusland was. \\

Na de vraag te stellen: "Zou je de boodschap in het schilderij herkennen nadat je het artikel hebt gelezen?" reageerde Gustas met: "Voor mij geeft niets aan dat het over een G7-bijeenkomst gaat." \\

De juiste boodschap werd niet uit dit voorbeeld gehaald.

\subsubsection{Sample 2}
Bij het tweede kunstwerk merkte Gustas op dat het zeker over gezondheidsproblemen ging. Hij vroeg zich af of er een nieuw medicijn was gekomen voor tijdens de zwangerschap. \\

Hoewel de boodschap gedeeltelijk juist was, was het specifieke medicijn eigenlijk bedoeld voor de menopauze, niet voor tijdens de zwangerschap. \\

Gustas bevestigde dat hij dit wel degelijk in het kunstwerk zag, maar erkende dat het zeer moeilijk is om de menopauze uit te beelden binnen een schilderij.

\subsubsection{Sample 3}
Het derde kunstwerk werd tot nu toe het best geïnterpreteerd door Gustas. Hij kon afleiden dat de Europese Unie op de een of andere manier verband hield met Facebook, maar wist niet precies hoe. Hij vermoedde dat er een nieuwe regel was opgelegd aan Facebook. \\

Gustas zat er heel dichtbij, maar net niet helemaal. Nadat hij het artikel had gezien, vroeg ik hem of hij nu de boodschap uit het schilderij kon halen. \\

Hij was hier niet zeker van omdat er niets was dat aanduidde dat het met geld te maken had.

\subsubsection{Sample 4}
In eerste instantie kon Gustas de gebeurtenis van 9/11 niet direct herkennen in het schilderij. Na enkele ogenblikken, werd het hem uiteindelijk duidelijk en kon hij de boodschap achter het kunstwerk ontcijferen.

\subsubsection{Sample 5}
Bij het laatste schilderij was de boodschap voor Gustas duidelijk, alhoewel hij aanvankelijk een vergissing maakte. Hij dacht dat het ging over een kerk die in brand was gestoken in Spanje, maar even later corrigeerde hij zichzelf en gaf toe dat hij de Notre-Dame in Frankrijk bedoelde. Ondanks de initiële verwarring, begreep Gustas uiteindelijk de boodschap van het schilderij. \\

Het is duidelijk dat de boodschap van beide schilderijen wel degelijk succesvol overkwam bij Gustas, zij het met enige vertraging in het geval van het tweede schilderij

\subsection{Persona: Shauny}
Shauny is een laatstejaarsstudente Journalistiek aan de Arteveldehogeschool met een grote interesse in politiek. Shauny volgt bijna dagelijks het nieuws en heeft enkele maanden bij een nieuwszender gewerkt als journalist. Ze lijkt me een perfecte persona voor deze turingtest.

\subsubsection{Sample 1}
Shauny merkte op: "Gaat dit over iets met Bakhmut? Het is moeilijk om te zeggen, ik heb geen idee wat die vlaggen of het wapenschild betekenen. Is het mogelijk een groep landen die samenkomen om iets te doen?" 

Shauny had grotendeels de kernboodschap begrepen uit dit artikel maar kon niet de link leggen met de G7-bijeenkomst.  \\

Na het tonen van de artikel, kon Shauny nu wel de betekenis hieruit halen. 

Shauny had grotendeels de essentiële boodschap begrepen uit dit artikel, maar kon de link met de G7-bijeenkomst niet leggen.

Nadat het artikel was getoond, kon Shauny nu wel de betekenis ervan afleiden.

"Als ik dit artikel had gezien of gelezen, dan zou ik waarschijnlijk aan zoiets hebben gedacht."

\subsubsection{Sample 2}
Shauny kon meteen afleiden uit het schilderij dat het ging over een vrouw die last had van haar buik. Ze dacht dat er mogelijk een probleem was met anticonceptiepillen. Echter, ze kon niet precies afleiden dat het eigenlijk ging om nieuwe medicatie voor de menopauze.\\

Na het tonen van het artikel kon Shauny echter wel duidelijk de betekenis ervan afleiden.

\subsubsection{Sample 3}
Bij deze sample dacht Shauny dat er gestemd werd over Griekenland in verband met een bepaalde nieuwe regel. Echter, nadat haar werd verteld dat de duimpjes symbool stonden voor Facebook, dacht ze dat er nieuwe regels voor Facebook werden opgesteld. \\

Toen Shauny het artikel te horen kreeg, vertelde ze me dat ze waarschijnlijk de boodschap hieruit niet kon afleiden. Ze gaf aan dat er enkele elementen ontbraken die aangaven dat het om een boete ging.

\subsubsection{Sample 4}
Shauny herkende onmiddellijk de associatie met de gebeurtenis van 9/11 bij het bekijken van dit schilderij. Haar interpretatie was correct en ze begreep direct de boodschap van de sample.

\subsubsection{Sample 5}
Net zoals bij de vorige sample kon Shauny vrijwel direct de boodschap uit dit schilderij halen. Haar vermogen om de betekenis ervan snel te begrijpen, toont aan dat de boodschap effectief werd overgebracht en begrepen werd door Shauny.

\subsection{Persona: Filip}
Filip is een busschauffeur met een passie voor schilderen en het maken van collages. Hij is een zeer creatieve persoon vol inspiratie en streeft ernaar om het nieuws dagelijks nauwlettend te volgen.

\subsubsection{Sample 1}
Bij het tonen van het eerste kunstwerk dacht Filip dat het ging over een bijeenkomst in Oekraïne, waar politici van verschillende landen samenkwamen met de president van Oekraïne om een plan te bedenken met betrekking tot Rusland. \\

Hoewel de boodschap grotendeels begrepen werd, kon de specifieke link naar het juiste artikel niet worden gelegd.

\subsubsection{Sample 2}
Het volgende schilderij werd iets moeilijker geïnterpreteerd door Filip. Hij meende dat de vrouw leed onder verkeerde medicatie en vermoedde dat het ging over situaties waarin artsen vaak verkeerde medicatie voorschrijven aan patiënten. \\

Na het vertellen dat het eigenlijk ging over nieuwe medicatie voor vrouwen tijdens de menopauze, kon hij deze boodschap wel duidelijk herkennen. Het kunstwerk kon echter met meerdere zaken in verband worden gebracht vertelde Filip.

\subsubsection{Sample 3}
Bij het 3e kunstwerk zag Filip verschillende elementen naar voren komen, zoals een televisie, de Europese Unie, het vind-ik-leuk-symbool en vragen. \\

Filip interpreteerde de boodschap van dit schilderij als volgt: "De Europese Unie heeft nieuwe regels opgelegd, waarbij mensen in het parlement het eens waren of vragen hadden. Maar de inwoners en televisiekijkers hadden gemengde gevoelens hierover." \\

Ondanks de inspiratie en creativiteit van Filip werd de boodschap volledig anders geïnterpreteerd. Nadat het artikel aan Filip werd voorgelezen, gaf hij aan dat deze boodschap te abstract was om eruit te halen. Misschien als hij het "vind-ik-leuk"-symbool had geïnterpreteerd als Facebook, zou het beter gelukt zijn.

\subsubsection{Sample 4}
Filip kon onmiddellijk de boodschap achterhalen bij de sample die verwees naar de 9/11-gebeurtenis. Voor hem was deze sample zeer duidelijk en begrijpelijk.

\subsubsection{Sample 5}
Na even twijfel wist Filip mij te vertellen dat het schilderij betrekking had op de brandende kerk in Parijs, de Notre-Dame. Ook bij deze sample slaagde Filip erin de boodschap en betekenis uit het schilderij te halen.


\subsection{Persona: Marta}
Marta is een kinderjuffrouw die geen artistieke bezigheden heeft en zichzelf niet als creatief beschouwt. Ze volgt echter dagelijks het nieuws.

\subsubsection{Sample 1}
Marta interpreteert het volgende: "Dit heeft duidelijk iets met Oekraïne te maken, maar ik weet niet precies wat. De verschillende vlaggen die ik niet herken, suggereren mogelijk de betrokkenheid van verschillende landen." \\

Marta kon het artikel niet direct aan dit kunstwerk linken, maar begreep wel de kernboodschap ervan. Toen ik Marta vroeg of ze het artikel herkende, vertelde ze me dat Oekraïne dagelijks in het nieuws komt en het moeilijk is om een specifiek artikel daaraan te koppelen, vooral als het kunstwerk algemeen is, naar mijn mening.

\subsubsection{Sample 2}
"Gaat dit over een nieuw medicijn tijdens de menopauze?" merkte Marta op. \\

Marta kon direct het artikel hiermee in verband brengen en de boodschap begrijpen.

\subsubsection{Sample 3}
"Hmm, dit is moeilijk. Het heeft iets te maken met de Europese Unie en verschillende meningen worden eraan gekoppeld. Misschien gaat het over een nieuwe regel die is opgelegd en waar mensen uiteenlopende standpunten over hebben? Nee, ik kan het niet echt begrijpen", zei Marta. \\

Nadat ik vertelde dat het artikel gaat over een recordboete die aan Facebook is opgelegd wegens schending van privacyregels, kon Marta het er niet in herkennen. Het onderwerp is te abstract voor haar.

\subsubsection{Sample 4}
Direct na het bekijken van het schilderij kon Marta met zekerheid vaststellen dat het de WTC-torens in Amerika betrof. Ze legde de associatie met 9/11 door het aanwezige vliegtuig in de buurt van de torens.

\subsubsection{Sample 5}
Bij haar eerste indruk dacht Marta dat het een afbeelding was van een brandende kerk. Ze kon niet onmiddellijk identificeren dat het specifiek de Notre-Dame betrof. Na een hint kon ze echter wel de juiste connectie maken en begreep ze dat het schilderij inderdaad verwijst naar de Notre-Dame. Ondanks de initiële verwarring slaagde Marta er uiteindelijk in om de juiste boodschap van het schilderij te begrijpen.