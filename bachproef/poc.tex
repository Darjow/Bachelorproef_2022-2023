\chapter{\IfLanguageName{dutch}{proof-of-concept}{proof-of-concept}}%
\label{ch:proof-of-concept}
\section{Inleiding}
In dit hoofdstuk bespreken we x en y waarin we voor elke fase een korte implementatie toelichten en de resultaat hiervan weergeven om zo met steeds een output naar de volgende fase te kunnen gaan ... 

\section{Webscraper}
\subsection{Algemene Eisen}
Titels, frontpage, eerste X ...
\subsection{Implementatie}
korte code stukken met toelichtingen + referentie naar literatuur studie (ethische aspecten, sleep toevoegen als hij href opent, ... )
\subsection{Uitkomst}
Deze uitkomst zal gebruikt worden binnen volgende fase ...  \\

\section{GPT}
\subsection{Inleiding}
Praten over prompten
\subsection{Implementatie}
korte code stukken met toelichtingen
\subsection{Uitkomst}
vertellen wat de uitkomst is van de prompts, en dat we dit nu kunnen gebruiken om een schilderij te genereren.

\section{DALL-E}
\subsection{Inleiding}
Kort nog eens de moeilijkheden uitleggen (interpretatie etc.)
\subsection{Implementatie}
Kort uitleggen
\subsection{Resultaat}
...
\begin{listing}[H]
\begin{minted}[breaklines, style=solarized-dark]{python}
   # This is a random comment
   class Yo:
        def fn:
            return 1337
   
  print(``yo'')
  
\end{minted}
\caption{Test code format}
\end{listing}