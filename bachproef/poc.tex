\chapter{\IfLanguageName{dutch}{proof-of-concept}{proof-of-concept}}%
\label{ch:proof-of-concept}
\section{Inleiding}
In dit hoofdstuk bespreken we de implementatie van het proof-of-concept. We zullen elke fase van het proces in detail bespreken en de implementatie van de bijbehorende componenten toelichten. Hierbij zullen we ook de uitkomsten van elke fase beschrijven en hoe deze kunnen gebruikt worden in de volgende fase. \\

\section{Webscraper}
\subsection{Algemene Eisen}
\ref{subsection:scraper-ethische-aspecten}
Titels, frontpage, eerste X ...
\subsection{Implementatie}
korte code stukken met toelichtingen + referentie naar literatuur studie (ethische aspecten, sleep toevoegen als hij href opent, ... )
\subsection{Uitkomst}
Deze uitkomst zal gebruikt worden binnen volgende fase ...  \\

\section{GPT}
\subsection{Inleiding}
Praten over prompten
\subsection{Implementatie}
korte code stukken met toelichtingen
\subsection{Uitkomst}
vertellen wat de uitkomst is van de prompts, en dat we dit nu kunnen gebruiken om een schilderij te genereren.

\section{DALL-E}
\subsection{Inleiding}
Kort nog eens de moeilijkheden uitleggen (interpretatie etc.)
\subsection{Implementatie}
Kort uitleggen
\subsection{Resultaat}
...
\begin{listing}[H]
\begin{minted}[breaklines, style=solarized-dark]{python}
   # This is a random comment
   class Yo:
        def fn:
            return 1337
   
  print(``yo'')
  
\end{minted}
\caption{Test code format}
\end{listing}