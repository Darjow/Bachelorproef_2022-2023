%---------- Inleiding ---------------------------------------------------------

\section{Introductie}%
\label{sec:introductie}
\noindent
TODO, onderwerp wat aankondigen en probleem of meerwaarde  eraan koppelen. \\
%In een wereld van digitalisatie met verschillende nieuwswebsites en social-media platformen kost het soms wat tijd om up-to-date te zijn met wat er dagelijks allemaal gebeurd.
%Om de tijd die het nodig is om informatie terug te vinden op de allerhande websites en social-media platformen in te korten, is er een nood aan een geautomatiseerde oplossing. \\
\noindent
Binnen mijn batchelorproef zal ik een toegepast onderzoek uitvoeren die een kunstwerk zal genereren met behulp van één of meerdere deep learning model(s).
Dit kunstwerk zal een visualisatie zijn van wat er die dag het hoogtepunt was in het nieuws en social media.


%---------- Stand van zaken ---------------------------------------------------

\section{Literatuurstudie}%
 TODO



%---------- Methodologie ------------------------------------------------------
\section{Methodologie}%
\label{sec:methodologie}
\noindent
\textbf{Inleiding} \\
De batchelorproef begint 2 maart 2023 en zal beëindigt worden voor 28 mei 2023. \\

\noindent
\textbf{Fase 1: Realiseren van een scraper} \\
Om de data te bekomen van de verschillende soorten websites en social-media platformen zal er een web scraper worden gemaakt. Deze scraper zal ontwikkeld worden in python met behulp van een externe library \emph{BeautifulSoup}.

\noindent
Doordat de presentatie van de verschillende artikelen kunnen verschillen in taal en structuur, zal de scraper een algoritme implementeren die het mogelijk maakt om op een uniforme manier verschillende websites te scrapen. \\

\noindent
\textbf{Fase 2: Data verwerken en analyseren} \\
Hier zullen we achterhalen op welke manier we de bekomen data uit voorgaande fase kunnen analyseren en verwerken. Dit zal dan ook geïmplementeerd worden zodat we steeds het belangrijkste artikel eruit kunnen halen. \\

\noindent
\textbf{Fase 3: Kunstwerk genereren} \\
Nu dat we de weten uit de vorige fase wat de hoogtepunt van de dag was. Kunnen we hierop een kunstwerk laten genereren. \\
Hiervoor zal er gebruik gemaakt worden van (een) deep learning model(s)  Dall-E en/of Stable Diffusion die de kern tekst van het artikel zal omvormen tot een foto. \\

\noindent
Één of beide technologieën zullen gebruikt worden binnenin python. \\

\noindent
\textbf{Fase 4: Twitter bot} \\ 
Nu dat we alle informatie hebben dat nodig is, kunnen we hiervoor een Twitter bot ontwikkelen. 
Allereerst is er een backend nodig, hier zal er voor node.js gekozen omdat deze ook perfect integreert met de API van Twitter.
Deze backend zal online geplaatst worden met behulp van Heroku. \\

\noindent
De backend zal 2 taken hebben om te verrichten:
\begin{itemize}
    \item De gegenereerde foto uit de vorige fase ontvangen
    \item De gegenereerde foto op een bepaald tijdstip tweeten in een interval.
\end{itemize}


%---------- Verwachte resultaten ----------------------------------------------
\section{Verwacht resultaat, conclusie}%
\label{sec:verwachte_resultaten}
TODO
